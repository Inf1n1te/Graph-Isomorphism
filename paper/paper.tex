\documentclass[twoside]{article}

\usepackage{amsmath,amssymb,amsthm} % Mathematical Symbols, styles, etc

\usepackage[utf8]{inputenc} % UTF-8 character encoding stuff

% Used to enable use of the defined labeled enunciations. Usage: \begin{definition}text\end{definition}
\newtheorem{theorem}{Theorem}[section]
\newtheorem{lemma}[theorem]{Lemma}
\newtheorem{proposition}[theorem]{Proposition}
\newtheorem{corollary}[theorem]{Corollary}

% --Optional-- Font and tweaking used in template
\usepackage[sc]{mathpazo} % Use the Palatino font
\usepackage[T1]{fontenc} % Use 8-bit encoding that has 256 glyphs
\linespread{1.05} % Line spacing - Palatino needs more space between lines
\usepackage{microtype} % Slightly tweak font spacing for aesthetics

% Formatting, refs and stuff
\usepackage[hmarginratio=1:1,top=32mm,columnsep=20pt]{geometry} % Document margins
\usepackage{multicol} % Used for the two-column layout of the document
\usepackage[hang, small,labelfont=bf,up,textfont=it,up]{caption} % Custom captions under/above floats in tables or figures
\usepackage{booktabs} % Horizontal rules in tables
\usepackage{float} % Required for tables and figures in the multi-column environment - they need to be placed in specific locations with the [H] (e.g. \begin{table}[H])
\usepackage{hyperref} % For hyperlinks in the PDF

% Fancy lists. Usage: http://lmgtfy.com/?q=latex+paralist
\usepackage{paralist} % Used for the compactitem environment which makes bullet points with less space between them

% Abstract customization
\usepackage{abstract} % Allows abstract customization
\renewcommand{\abstractnamefont}{\normalfont\bfseries} % Set the "Abstract" text to bold
\renewcommand{\abstracttextfont}{\normalfont\small\itshape} % Set the abstract itself to small italic text

% Title customization
\usepackage{titlesec} % Allows customization of titles
%\renewcommand\thesection{\Roman{section}} % Roman numerals for the sections
%\renewcommand\thesubsection{\Roman{subsection}} % Roman numerals for subsections
\titleformat{\section}[block]{\large\scshape}{\thesection.}{1em}{} % Change the look of the section titles
\titleformat{\subsection}[block]{\large}{\thesubsection.}{1em}{} % Change the look of the section titles

% Header and footer customization
\usepackage{fancyhdr} % Headers and footers
\pagestyle{fancy} % All pages have headers and footers
\fancyhead{} % Blank out the default header
\fancyfoot{} % Blank out the default footer
\fancyhead[C]{T.\ Kemp, T.\ Kerkhoven, J.\ Klein Brinke, T.\ Sonderen: \shorttitle} % Custom header text
\fancyfoot[RO,LE]{\thepage} % Custom footer text

% Bibliography
\usepackage[backend=bibtex]{biblatex}
\bibliography{references.bib}

%----------------------------------------------------------------------------------------
%	TITLE SECTION
%----------------------------------------------------------------------------------------

\newcommand{\articletitle}{The Effects of Preprocessing Methods on Computation Times for Graph Isomorphisms}
\newcommand{\shorttitle}{Preprocessing for Graph Isomorphisms}

\title{\vspace{-15mm}\fontsize{24pt}{10pt}\selectfont\textbf{\articletitle}} % Article title

\author{
\large
\textsc{Tim Kemp, Tim Kerkhoven, Jeroen Klein Brinke, and Tim Sonderen}\\[2mm] % Your names
\normalsize University of Twente \\ % Your institution
\normalsize \href{mailto:t.kerkhoven@student.utwente.nl}{t.kerkhoven@student.utwente.nl}, 
\href{mailto:t.kemp@student.utwente.nl}{t.kemp@student.utwente.nl},\\ \normalsize
\href{mailto:j.klein.brinke@student.utwente.nl}{j.klein.brinke@student.utwente.nl}, \href{mailto:t.sonderen@student.utwente.nl}{t.sonderen@student.utwente.nl}% Your email addresses
}

\date{\today}

%----------------------------------------------------------------------------------------

\begin{document}

\thispagestyle{empty}
\maketitle % Insert title

%----------------------------------------------------------------------------------------
%	ABSTRACT
%----------------------------------------------------------------------------------------

\begin{abstract}

\noindent Abstract here

\end{abstract}

%----------------------------------------------------------------------------------------
%	ARTICLE CONTENTS
%----------------------------------------------------------------------------------------

\begin{multicols}{2} % Two-column layout throughout the main article text

%------------------------------------------------
\section{Introduction} 

This paper will discuss the effect of several preprocessing methods on the computation time of the graph isomorphism problem. This paper will mostly focus on the algorithms and their correctness, in combination with computational experiments and their results.

The graph isomorphism problem is the computational problem of determining whether two finite graphs are isomorphic. Besides its practical importance, the graph isomorphism problem is a curiosity in computational complexity theory as it is one of a very small number of problems belonging to NP neither known to be solvable in polynomial time nor NP-complete: it is one of only 12 such problems listed by Garey \& Johnson (1979)\cite{book:garyJohnson1979}, and one of only two of that list whose complexity remains unresolved. As of 2008 the best algorithm (Eugene Luks, 1983) has run time $2^{O(\sqrt{n log n})}$ for graphs with \emph{n} vertices.\cite{article:davidJohnson2005}\cite{inproceedings:babaiCodenotti2008}\cite{website:wikiGI}

The preprocessing part are algorithms that look at the graphs before determining whether they are isomorphic with the intent of speeding up the actual process of the determining whether the graphs are isomorphic.

%------------------------------------------------

\section{Scope \& Objectives}

The aim of this research paper is to determine whether certain preprocessing methods significantly reduce the time required to check whether two graphs are isomorphic. At the core, two cases will be compared: a basic case without preprocessing, and a case with preprocessing. 

For this paper, the time complexity of the algorithms will not be discussed or calculated. The difference in performance of the cases will be shown from a statistical analysis of computational results.

This paper will give a description of the algorithms used for preproccesing, as well as proof of correctness. The algorithms used to check for isomorphisms will be left out of consideration.

%------------------------------------------------
\section{Methods}

TBD - exact setup of experiment - what methods were used to test
Used to examine validity of the experiment by outsiders, by very precise, give reasons for methods and setup, possibly give alternatives and arguments for them, but describe why we chose the current methods.

%------------------------------------------------
\section{Description of Algorithms}

Description of Algorithms here
Don't forget to reference the slides

%------------------------------------------------
\section{Proof of Correctness}

\begin{definition}
Let $G_1 = (V_1, E_1)$ and $G_2 = (V_2, E_2)$ be two undirected colored graphs. A function $f: V_1 \to V_2$ is called a \textit{graph isomorphism} if (a) $f$ is a bijection, (b) for all $a, b \in V_1, \{a, b\} \in E_1$ if and only if $\{f(a), f(b)\} \in E_2$, and (c) $color(v)$ = $color (f(v)$. When such a function exist, $G_1$ and $G_2$ are called \textit{ismorphic graphs}.
\end{definition}
\begin{definition}
Let $G = (V,E)$ be a graph. The \textit{neighbourhood} $N(v)$ of $v \in V$ is:
$N(v) = \left\{{u \in V : \{u,v\} \in E}\right\}$
\end{definition}
\begin{definition}
Let $G = (V,E)$ be a graph.
Two vertices $u, v \in V$ are \textit{twins} if $\{u,v\} \in E$ and $N(u)\textbackslash {v} = N(v)\textbackslash {u}$.
\end{definition}
\begin{definition}
Let $G = (V,E)$ be a graph.
Two vertices $u, v \in V(G)$ are \textit{false twins} if $N(u) = N(v).$
\end{definition}

%------------------------------------------------
\section{Computational Experiments}

Computational Experiments here
Setup good experiments beforehand and test extensively

%------------------------------------------------
\section{Results}

Results here
And statistical analysis of the results

%------------------------------------------------
\section{Conclusions}

Conclusions here 
Is it faster?

%----------------------------------------------------------------------------------------
%	REFERENCE LIST
%----------------------------------------------------------------------------------------

\printbibliography

%----------------------------------------------------------------------------------------

\end{multicols}

\end{document}
