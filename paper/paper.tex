\documentclass[twoside]{article}

\usepackage{amsmath,amssymb,amsthm} % Mathematical Symbols, styles, etc

% Used to enable use of the defined labeled enunciations. Usage: \begin{definition}text\end{definition}
\newtheorem{theorem}{Theorem}[section]
\newtheorem{lemma}[theorem]{Lemma}
\newtheorem{proposition}[theorem]{Proposition}
\newtheorem{corollary}[theorem]{Corollary}

% --Optional-- Font and tweaking used in template
\usepackage[sc]{mathpazo} % Use the Palatino font
\usepackage[T1]{fontenc} % Use 8-bit encoding that has 256 glyphs
\linespread{1.05} % Line spacing - Palatino needs more space between lines
\usepackage{microtype} % Slightly tweak font spacing for aesthetics

% Formatting, refs and stuff
\usepackage[hmarginratio=1:1,top=32mm,columnsep=20pt]{geometry} % Document margins
\usepackage{multicol} % Used for the two-column layout of the document
\usepackage[hang, small,labelfont=bf,up,textfont=it,up]{caption} % Custom captions under/above floats in tables or figures
\usepackage{booktabs} % Horizontal rules in tables
\usepackage{float} % Required for tables and figures in the multi-column environment - they need to be placed in specific locations with the [H] (e.g. \begin{table}[H])
\usepackage{hyperref} % For hyperlinks in the PDF

% Fancy lists. Usage: http://lmgtfy.com/?q=latex+paralist
\usepackage{paralist} % Used for the compactitem environment which makes bullet points with less space between them

% Abstract customization
\usepackage{abstract} % Allows abstract customization
\renewcommand{\abstractnamefont}{\normalfont\bfseries} % Set the "Abstract" text to bold
\renewcommand{\abstracttextfont}{\normalfont\small\itshape} % Set the abstract itself to small italic text

% Title customization
\usepackage{titlesec} % Allows customization of titles
%\renewcommand\thesection{\Roman{section}} % Roman numerals for the sections
%\renewcommand\thesubsection{\Roman{subsection}} % Roman numerals for subsections
\titleformat{\section}[block]{\large\scshape}{\thesection.}{1em}{} % Change the look of the section titles
\titleformat{\subsection}[block]{\large}{\thesubsection.}{1em}{} % Change the look of the section titles

% Header and footer customization
\usepackage{fancyhdr} % Headers and footers
\pagestyle{fancy} % All pages have headers and footers
\fancyhead{} % Blank out the default header
\fancyfoot{} % Blank out the default footer
\fancyhead[C]{T.\ Kemp, T.\ Kerkhoven, J.\ Klein Brinke, T.\ Sonderen: \shorttitle} % Custom header text
\fancyfoot[RO,LE]{\thepage} % Custom footer text

%----------------------------------------------------------------------------------------
%	TITLE SECTION
%----------------------------------------------------------------------------------------

\newcommand{\articletitle}{The Effects of Preprocessing Methods on Computation Times for Graph Isomorphisms}
\newcommand{\shorttitle}{Preprocessing for Graph Isomorphisms}

\title{\vspace{-15mm}\fontsize{24pt}{10pt}\selectfont\textbf{\articletitle}} % Article title

\author{
\large
\textsc{Tim Kemp, Tim Kerkhoven, Jeroen Klein Brinke, and Tim Sonderen}\\[2mm] % Your names
\normalsize University of Twente \\ % Your institution
\normalsize \href{mailto:t.kerkhoven@student.utwente.nl}{t.kerkhoven@student.utwente.nl}, 
\href{mailto:t.kemp@student.utwente.nl}{t.kemp@student.utwente.nl},\\ \normalsize
\href{mailto:j.klein.brinke@student.utwente.nl}{j.klein.brinke@student.utwente.nl}, \href{mailto:t.sonderen@student.utwente.nl}{t.sonderen@student.utwente.nl}% Your email addresses
}

\date{\today}

%----------------------------------------------------------------------------------------

\begin{document}

\thispagestyle{empty}
\maketitle % Insert title

%----------------------------------------------------------------------------------------
%	ABSTRACT
%----------------------------------------------------------------------------------------

\begin{abstract}

\noindent Abstract here

\end{abstract}

%----------------------------------------------------------------------------------------
%	ARTICLE CONTENTS
%----------------------------------------------------------------------------------------

\begin{multicols}{2} % Two-column layout throughout the main article text

%------------------------------------------------
\section{Introduction}

Introduction here

%------------------------------------------------

\section{Scope \& Objectives}

Scope \& Objectives here

%------------------------------------------------
\section{Methods}

Methods here

%------------------------------------------------
\section{Description of Algorithms}

Description of Algorithms here

%------------------------------------------------
\section{Proof of Correctness}

Proof of Correctness here

%------------------------------------------------
\section{Computational Experiments}

Computational Experiments here

%------------------------------------------------
\section{Results}

Results here

%------------------------------------------------
\section{Conclusions}

Conclusions here 

%----------------------------------------------------------------------------------------
%	REFERENCE LIST
%----------------------------------------------------------------------------------------
 
% Bibliography - this is intentionally simple in this template
\begin{thebibliography}{99}

\bibitem{Fig2009}
Figueredo, A.~J.\ and Wolf, P.\ S.~A.\ (2009).
Assortative pairing and life history strategy - a cross-cultural study.
\emph{Human Nature \& Arts}, 20:317-330.

\bibitem{Lang2011}
R.\ Langerak  (2011).
How to write a {\tt python} computer program.
\emph{Annals of Improbable Research}, 2:16-27.

\end{thebibliography}

%----------------------------------------------------------------------------------------

\end{multicols}

\end{document}
